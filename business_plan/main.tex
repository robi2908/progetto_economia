\documentclass[12pt, a4paper]{article}

\usepackage[]{graphicx}

% FONT
\usepackage{newtxtext}

\usepackage[margin=3cm]{geometry}

% interlinea
\usepackage{setspace}
\renewcommand{\baselinestretch}{1.5} 

\newcommand{\meskip}{\medskip \\}

\usepackage{enumitem}

\title{
    \includegraphics[width=.8\textwidth]{images/LOGO.png}\\
    \textbf{Business Plan}
}
\date{}
\author{}

\begin{document}

\maketitle

\newpage

\tableofcontents

\newpage

\section{Descrizione dell'impresa}
La nostra startup è stata fondata con un obiettivo chiaro: offrire un'esperienza enologica eccezionale a tutti gli appassionati del vino.
Siamo fermamente convinti che il vino non sia solo una bevanda, ma un'arte che merita di essere scoperta e apprezzata appieno.\meskip
Attraverso la nostra piattaforma online, abbiamo creato un ambiente virtuale in cui i nostri clienti possono immergersi nel meraviglioso mondo del vino. Offriamo loro l'opportunità di esplorare un vasto assortimento di vini provenienti da diverse regioni e cantine, consentendo loro di ampliare le loro conoscenze e scoprire nuovi gusti. Inoltre, forniamo recensioni autentiche e affidabili, garantendo ai nostri clienti una guida preziosa nella scelta del vino più adatto alle loro preferenze.\meskip
Sappiamo che la comodità è un elemento fondamentale nella vita di oggi, quindi ci assicuriamo che i nostri clienti possano godere di un'esperienza di acquisto senza problemi. Grazie al nostro sistema di ordini online, possono selezionare i loro vini preferiti e riceverli comodamente a casa propria. E per garantire che ogni bottiglia arrivi in condizioni ottimali, abbiamo stretto partnership con servizi di spedizione veloci e affidabili.\meskip
Ma c'è qualcosa che ci rende davvero speciali.
In Vinovo, mettiamo un'enfasi particolare sulla sostenibilità nel settore vinicolo.
Siamo consapevoli dell'importanza di preservare l'ambiente e lavoriamo solo con produttori che adottano pratiche ecologiche nella produzione dei loro vini.
Ci impegniamo a ridurre l'impatto ambientale, promuovendo l'uso responsabile delle risorse e incoraggiando l'agricoltura sostenibile.\meskip
In sintesi, Vinovo rappresenta molto più di una semplice piattaforma di vendita di vini. Rappresenta un'esperienza enologica completa, in cui la qualità e la sostenibilità si uniscono per offrire ai nostri clienti un viaggio indimenticabile nel mondo del vino. Vogliamo essere il punto di riferimento per tutti coloro che desiderano esplorare l'arte e il piacere di un buon bicchiere di vino, garantendo loro un accesso facile a una selezione vasta e curata.
Unisciti a noi in questo viaggio e lasciati conquistare dalla passione e dall'eccellenza che si cela dietro ogni nostra bottiglia di vino.

\subsection{Vision e mission}
\subsubsection*{Vision}
La nostra visione è quella di trasformare l'esperienza enologica in un viaggio affascinante e coinvolgente, in cui la qualità, la sostenibilità e la scoperta si fondono armoniosamente.
Vogliamo che ogni appassionato del vino possa esplorare e apprezzare i tesori enogastronomici del mondo, guidato dalla curiosità e dalla consapevolezza di fare scelte che rispettino l'ambiente.

\subsubsection*{Mission}
La nostra missione è quella di creare un legame forte tra i produttori di vini di alta qualità e i consumatori desiderosi di vivere un'esperienza enologica autentica.\\
Ci concentriamo sulla qualità dei prodotti, sulla sostenibilità delle pratiche produttive e sulla garanzia di un servizio di spedizione veloce e sicuro.\meskip
Ci adoperiamo attivamente per promuovere la sostenibilità nel settore vinicolo; vogliamo educare i consumatori sull'importanza di fare scelte sostenibili e offrire loro una selezione di vini che rispecchi questi valori.\meskip
Siamo determinati a diventare un punto di riferimento per tutti coloro che cercano\\ un'esperienza enologica autentica, basata sulla scoperta, sulla qualità e sulla sostenibilità.
Vogliamo creare una comunità in cui gli appassionati del vino possano condividere le loro esperienze, imparare dagli esperti del settore e lasciarsi ispirare da nuove tendenze e scoperte enogastronomiche.\meskip
In sintesi, la nostra visione è quella di offrire un'esperienza enologica di qualità, in cui la passione per il vino si unisce alla consapevolezza ambientale.

\subsection{Collaborazioni}
I nostri collaboratori sono una parte fondamentale del successo di Vinovo.
Collaboriamo con una rete diversificata di partner, tra cui ristoranti e locali, produttori di vini e aziende di logistica, che contribuiscono a rendere possibile la nostra missione.\meskip
I ristoranti e i locali con cui collaboriamo sono selezionati con cura per offrire ai nostri clienti un'esperienza culinaria raffinata e in sintonia con i vini di alta qualità presenti sulla nostra piattaforma.
Riconosciamo l'importanza di abbinare il vino ad una cucina di qualità, e la collaborazione con questi partner ci consente di offrire ai nostri clienti un'esperienza completa, in cui i sapori si armonizzano perfettamente.\meskip
Collaboriamo con cantine che mettono passione e impegno nella produzione di vini di alta qualità, rispettando l'ambiente.
Questa collaborazione ci consente di offrire ai nostri clienti una selezione accurata di vini unici e di raccontare le storie dietro ogni etichetta.\meskip
Le aziende di logistica con cui ci associamo sono responsabili della spedizione dei nostri vini.
Siamo consapevoli dell'importanza di garantire che ogni bottiglia arrivi in condizioni ottimali e nel minor tempo possibile.
Le aziende di logistica con cui collaboriamo condividono i nostri alti standard di affidabilità e sicurezza, garantendo che i nostri clienti possano godere dei loro vini preferiti senza preoccupazioni.\meskip
Riconosciamo il valore delle relazioni di collaborazione e crediamo nell'importanza di costruire partnership solide e durature.\meskip
Continueremo a coltivare queste partnership e a cercare nuove opportunità di collaborazione, perché crediamo che insieme possiamo offrire ai nostri clienti un viaggio indimenticabile nel mondo del vino.\newpage

\subsection{Bozza del sito web}
\begin{center}
    \frame{\includegraphics[width=.93\textwidth]{images/sito.png}}
\end{center}


\section{Analisi di mercato}

Il settore vinicolo italiano è un pilastro dell'economia, generando un notevole fatturato annuo di 14,5 miliardi di euro. Da solo, l'export di vino vale 7,1 miliardi di euro, registrando una significativa crescita del 12,4\% nel 2021. Inoltre, la bilancia commerciale dimostra un saldo positivo di circa 6,7 miliardi di euro, confermando l'importanza strategica di questo mercato.

\subsection{Dimensione del mercato}
Il consumo di vino in Italia coinvolge un vasto numero di persone, contando circa 30 milioni di consumatori, che rappresentano il 54\% della popolazione adulta del paese. Questa base di consumatori comprende sia bevitori occasionali che regolari, sottolineando l'ampia diffusione del vino italiano.\\
Un dato interessante è che, attualmente, il 66\% dei consumatori di vino sono uomini. Tuttavia, è degno di nota il fatto che il segmento delle donne sia in forte crescita durante questo decennio, con un aumento del 2,3\%. Le donne stanno assumendo un ruolo sempre più rilevante nel mercato del vino italiano, contribuendo alla sua evoluzione e alla sua diversificazione.\\
Per avere una visione completa del consumo di vino nel paese, si può fare riferimento al grafico che illustra la distribuzione del consumo nelle diverse regioni.
\begin{center}
    \includegraphics[width=.62\textwidth]{images/grafico_consumatori_regioni.png}
\end{center}
\noindent Dall'analisi del grafico sottostante emerge chiaramente un notevole aumento del consumo di vino da parte dei giovani. Questa tendenza rappresenta un'opportunità significativa per il settore vinicolo, in quanto indica un cambiamento nelle preferenze e nei comportamenti dei consumatori più giovani.\meskip
L'aumento del consumo di vino tra i giovani suggerisce un crescente interesse per la cultura enologica, la scoperta di nuovi sapori e l'esperienza di bevande di alta qualità. Tale tendenza potrebbe essere attribuita a una maggiore consapevolezza dei giovani sui benefici del vino, come le proprietà antiossidanti e la sua associazione con uno stile di vita sofisticato.\meskip
È quindi essenziale adattarsi a questa tendenza, offrendo una selezione di vini adatta ai gusti e alle preferenze dei giovani, utilizzando strategie di marketing mirate e coinvolgenti per catturare l'attenzione di questo segmento di mercato in crescita.
\begin{center}
    \includegraphics[width=\textwidth]{images/consumo_vino_eta.png}
\end{center}
La fascia di consumatori compresa tra i 30 e i 45 anni manifesta un forte interesse per tutti gli aspetti della sostenibilità, inclusi quelli ambientali, economici e sociali. Questi consumatori attribuiscono grande importanza alla provenienza etica dei vini, alle pratiche agricole sostenibili e alla responsabilità sociale delle aziende vinicole. La sostenibilità diventa quindi un fattore decisivo nella scelta dei vini per questa fascia di età.\meskip
D'altra parte, i giovani compresi tra i 18 e i 25 anni mostrano una particolare attrazione verso i vini di prestigio e riconosciuti a livello di brand. Tuttavia, sono anche alla ricerca di nuove esperienze e sono fortemente influenzati dagli eventi e dalle fiere del settore. Questa fascia di consumatori è attiva sui social media, dove seguono influencer del settore vinicolo e cercano informazioni su nuovi vini. È interessante notare che i giovani consumatori, se colpiti positivamente da un vino, tendono a voler approfondire l'esperienza, visitando direttamente le cantine.\meskip
Le donne, in particolare nella fascia di età compresa tra i 30 e i 45 anni, rappresentano oltre il 40\% degli iscritti a degustazioni e corsi di assaggio, ma sono anche la fascia di età che maggiormente apprezza e partecipa a visite aziendali e si avvale dei servizi legati all'Enoturismo.\meskip
Le donne hanno dimostrato una partecipazione attiva nel mondo del vino, sia in termini di formazione sia in termini di esperienze pratiche. Sono sempre più coinvolte in degustazioni guidate, corsi di assaggio e percorsi formativi per approfondire le loro conoscenze sul vino.

\subsection{Analisi PEST}

\subsubsection*{Fattori politici}
Vinovo opera in un'area geografica limitata all'Italia, il che presenta un vantaggio in quanto deve rispettare solo la legislazione nazionale.
La vendita di prodotti online non è soggetta a restrizioni significative, il che semplifica l'ingresso nel mercato, ma allo stesso tempo genera preoccupazione per la facilità con cui nuovi concorrenti potrebbero guadagnare quote di mercato.\\
In Italia, i siti web specializzati nella vendita online devono seguire una specifica normativa fiscale:
\begin{itemize}[itemsep=-5pt, topsep=0pt]
    \item aprire una partita IVA registrandosi presso la Camera di Commercio
    \item le informazioni dell'azienda devono essere chiaramente visibili sul portale online, consentendo agli utenti di trovare facilmente dati come il numero di partita IVA, l'indirizzo della sede legale e il codice di registrazione al Registro delle Imprese.
    \item si deve utilizzare il sistema di fatturazione elettronica, che prevede l'archiviazione digitale delle fatture per un periodo di 10 anni.
\end{itemize}

\subsubsection*{Fattori economici}
È importante tenere in considerazione il cambiamento climatico il quali potrebbe portare a difficoltà e quindi una riduzione significativa della produzione di uva in Italia.\\
Questa riduzione della quantità di uva disponibile potrebbe portare a un aumento dei costi di produzione per le aziende vinicole, in quanto l'offerta di uva è diminuita.\\
Se allo stesso tempo, supponiamo ci sia una crescente domanda di vini italiani all'estero, questo potrebbe portare a un aumento dei prezzi dei vini italiani sul mercato internazionale.
Di conseguenza, le aziende vinicole italiane potrebbero trovarsi ad affrontare costi di produzione più elevati a causa della riduzione dell'offerta di uva e potrebbero anche decidere di aumentare i prezzi dei loro vini per sfruttare la domanda crescente sui mercati internazionali.
Dunque è fondamentale instaurare forti collaborazioni con le aziende vinicole e soprattutto piccoli fornitori.

\subsubsection*{Fattori sociali}
Le preferenze dei consumatori italiani per il vino possono variare in base alle tradizioni regionali, alle abitudini alimentari e alle tendenze di consumo. Vinovo deve adattarsi alle preferenze dei clienti italiani e offrire una gamma di vini che risponda alle diverse esigenze e agli stili di vita dei consumatori locali.\\
Inoltre l'Italia è nota per la sua cultura enogastronomica, e Vinovo può sfruttare l'interesse dei consumatori per i vini italiani di qualità, promuovendo la provenienza, la tradizione e la varietà dei vini offerti.

\subsubsection*{Fattori tecnologici}
In Italia, l'innovazione tecnologica nel settore dell'e-commerce, come miglioramenti nella sicurezza dei pagamenti online, nell'esperienza di acquisto e nella logistica di consegna, influiscono sulla facilità con cui i consumatori possono acquistare vino online.
L'adozione di strumenti digitali e soluzioni tecnologiche nel settore vinicolo può influenzare la competitività della nostra proposta. Ad esempio, l'utilizzo di strumenti di analisi dei dati per comprendere meglio i comportamenti dei clienti e offrire raccomandazioni personalizzate può contribuire al successo di Vinovo.\bigskip
\begin{center}
    \includegraphics[width=.9\textwidth]{images/pic-pest-analysis.png}
\end{center}

\end{document}



